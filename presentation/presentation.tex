%!TEX program = xelatex
\documentclass{beamer}
\usepackage{blindtext}
\usepackage{color}
\definecolor{bluekeywords}{rgb}{0.13,0.13,1}
\definecolor{greencomments}{rgb}{0,0.5,0}
\definecolor{redstrings}{rgb}{0.9,0,0}
\definecolor{light-gray}{gray}{0.85}
\definecolor{white}{rgb}{1,1,1}
\usepackage{listings}
\lstset{language=[Sharp]C,
	showspaces=false,
	showtabs=false,
	numbers=left,
    breaklines=true,
    backgroundcolor=\color{white},
	showstringspaces=false,
	breakatwhitespace=true,
	escapeinside={(*@}{@*)},
	commentstyle=\color{greencomments},
	keywordstyle=\color{bluekeywords},
	stringstyle=\color{redstrings},
	basicstyle=\fontsize{6}{6}\selectfont\ttfamily,
	tabsize=3
}
\usetheme{Execushares}
\title{Unit testing EPiSERVER}
\subtitle{How to write testable and maintainable code}
\author{Tommy Ivarsson}
\date{\today}

\setcounter{showSlideNumbers}{1}

\begin{document}
	\setcounter{showProgressBar}{0}
	\setcounter{showSlideNumbers}{0}

	\frame{\titlepage}

	\setcounter{framenumber}{0}
	\setcounter{showProgressBar}{1}
	\setcounter{showSlideNumbers}{1}
	\section{Testable code}
		\begin{frame}
			\frametitle{What does it mean and why?} \pause
			\begin{exampleblock}{What does it mean to write testable code?} \pause
				\begin{itemize}
					\item Write modules which are not tightly bound to the implementation of any dependency \pause
					\item Know the limitations of your testing frameworks and write your code accordingly \pause
				\end{itemize}
			\end{exampleblock}
			\begin{exampleblock}{Why do we want testable code?} \pause
				\begin{itemize}
					\item Confidence \pause
					\begin{itemize}
						\item Confidence in not breaking the work of others \pause
						\item Confidence in not changing existing logic when adding new logic
					\end{itemize}
				\end{itemize}
			\end{exampleblock}
		\end{frame}
		\begin{frame}
			\frametitle{Difficult to test}
			\lstinputlisting{code-examples/intro-hard.cs}
		\end{frame}
		\begin{frame}
			\frametitle{Easy to test}
			\lstinputlisting{code-examples/intro-easy.cs}
		\end{frame}

	\section{What are the main properties of testable code?}
		\begin{frame}
			\frametitle{Properties of a testable method} \pause
			\begin{itemize}
				\item Pure function \pause
				\begin{itemize}
					\item $f(a) = b$ \pause
					\item No side effects \pause
				\end{itemize}
				\item Independent of application state \pause
				\item Independent of environment \pause
				\item Independent of dependency implementation \pause
				\item Dependencies are resolved using dependency injection
			\end{itemize}
		\end{frame}

	\section{What is dependency injection?}
		\begin{frame}
			\frametitle{Dependency injection} \pause
			\begin{exampleblock}{}
				{\large ``An injection is the passing of a dependency (a service) to a dependent object (a client).''}
				\vskip5mm
				\hspace*\fill{\small--- Wikipedia} \pause
			\end{exampleblock}
			\begin{exampleblock}{Advantages of Dependency Injection} \pause
				\begin{itemize}
					\item Code easier to test as dependencies can be mocked \pause
					\item Easily applied to existing code \pause
					\item Class dependencies listed in the constructor \pause
					\item Promotes reusability, testability and maintainability
				\end{itemize}
			\end{exampleblock}
		\end{frame}
		\begin{frame}
			\frametitle{Code without dependency injection}
			\lstinputlisting{code-examples/di-no-di.cs}
		\end{frame}
		\begin{frame}
			\frametitle{Code with dependency injection}
			\lstinputlisting{code-examples/di-di.cs}
		\end{frame}

	\section{Dependency injection in EPiSERVER}
		\begin{frame}
			\frametitle{Tempting, but not good}
			\lstinputlisting{code-examples/di-servicelocator.cs} \pause
			\begin{exampleblock}{Disadvantages} \pause
				\begin{itemize}
					\item Ugly \pause
					\item Requires more code for mock setup \pause
					\item Class dependencies not listed in constructor
				\end{itemize}
			\end{exampleblock}
		\end{frame}
		\begin{frame}
			\frametitle{Correct way to get an instance}
			\lstinputlisting{code-examples/di-servicelocator-ok.cs}
		\end{frame}
		\begin{frame}
			\frametitle{Registering your own service}
			\lstinputlisting{code-examples/di-structuremap.cs}
		\end{frame}

	\section{Unit testing frameworks}
	\begin{frame}
			\frametitle{Examples of unit testing frameworks} \pause
			\begin{exampleblock}{There are two main frameworks} \pause
			\begin{itemize}
				\item MS-Test \pause
				\begin{itemize}
					\item $+$ Included in Visual Studio \pause
				\end{itemize}
				\item NUnit \pause
				\begin{itemize}
					\item $+$ Code is more readable \pause
					\item $+$ Supports multiple test cases for the same test \pause
				\end{itemize}
			\end{itemize}
			\end{exampleblock}

			\begin{exampleblock}{I prefer NUnit but which one you use does not make a big difference}
			\end{exampleblock}
		\end{frame}

	\section{Mocking frameworks}
		% \begin{frame}
		% 	\frametitle{Why use a mocking framework?} \pause
		% 	\begin{exampleblock}{Why write code that you don't have to?} \pause
		% 		\begin{itemize}
		% 				\item Not using a mocking framework is silly \pause
		% 				\item Which one you use is up to you
		% 		\end{itemize}
		% 	\end{exampleblock}
		% \end{frame}

		\begin{frame}
			\frametitle{Examples of mocking frameworks} \pause
			\begin{exampleblock}{There are A LOT of frameworks to choose from, examples are:} \pause
			\begin{itemize}
				\item Moq \pause
				\begin{itemize}
					\item $+$ Popular in the EPiSERVER community \pause
					\item $+$ Lots of StackOverflow posts \pause
					\item $+$ Been around for ages \pause
					\item $-$ Lambda hell $\Rightarrow$ Difficult to read \pause
				\end{itemize}
				\item NSubstitute \pause
				\begin{itemize}
					\item $+$ Scarce use of lambdas $\Rightarrow$ Easy to read \pause
					\item $+$ Lots of StackOverflow posts \pause
					\item $-$ Not popular in the EPiSERVER community \pause
				\end{itemize}
			\end{itemize}
			\end{exampleblock}

			\begin{exampleblock}{I prefer NSubstitute but which one you use does not make a big difference}
			\end{exampleblock}
		\end{frame}

	\section{Writing tests}
		\begin{frame}
			\frametitle{Tools} \pause
			\begin{itemize}
				\item NUnit \pause
				\item NUnit Test Adapter \pause
				\item NSubstitute
			\end{itemize}
		\end{frame}
		\begin{frame}
			\frametitle{Methodologies} \pause
			\begin{itemize}
				\item Do not test external logic (for example EPiSERVER functionality)\pause
				\item One folder for each class, one test file for each method \pause
				\item Descriptive test names \pause
				\item Arrange, Act, Assert 
			\end{itemize}
		\end{frame}
		\begin{frame}
			\frametitle{Simple example}
			\lstinputlisting{code-examples/test-easy.cs}
		\end{frame}
		\begin{frame}
			\frametitle{TestCase example}
			\lstinputlisting{code-examples/test-easy2.cs}
		\end{frame}
		\begin{frame}
			\frametitle{Recall this method}
			\lstinputlisting{code-examples/intro-easy.cs}
		\end{frame}

		\begin{frame}
			\frametitle{The setup}
			\lstinputlisting{code-examples/test-nsubstitute-example.cs}
		\end{frame}

		\begin{frame}
			\frametitle{Arrange, Act, Assert}
			\lstinputlisting{code-examples/test-nsubstitute-example2.cs}
		\end{frame}

		\section{Unit testing in EPiSERVER}
		\begin{frame}
			\frametitle{Unit testing in EPiSERVER} \pause
			\begin{itemize}
				\item EPiSERVER has a history of being difficult to write unit tests for \pause
				\item Things have changed which has made things better - but not great \pause
				\begin{itemize}
					\item Heavy use of static methods (FilterForVisitor.Filter(), Category.Find() etc) \pause
					\item Heavy use of extension methods (almost anything in EPiSERVER Find) \pause
					\item Lack of interfaces
				\end{itemize}
			\end{itemize}
		\end{frame}

		\section{Conclusion}
		\begin{frame}
			\frametitle{A few simple pointers} \pause
			\begin{itemize}
				\item Test the class you are testing, not the dependencies \pause
				\item Use dependency injection properly \pause
				\item Isolate your logic \pause
				\item Programming is a craft. Care for your code like a proper craftsman 
			\end{itemize}
		\end{frame}
\end{document}

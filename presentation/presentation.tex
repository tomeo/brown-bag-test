%!TEX program = xelatex
\documentclass{beamer}
\usepackage{blindtext}
\usepackage{color}
\definecolor{bluekeywords}{rgb}{0.13,0.13,1}
\definecolor{greencomments}{rgb}{0,0.5,0}
\definecolor{redstrings}{rgb}{0.9,0,0}
\definecolor{light-gray}{gray}{0.85}
\definecolor{white}{rgb}{1,1,1}
\usepackage{listings}
\lstset{language=[Sharp]C,
	showspaces=false,
	showtabs=false,
	numbers=left,
    breaklines=true,
    backgroundcolor=\color{white},
	showstringspaces=false,
	breakatwhitespace=true,
	escapeinside={(*@}{@*)},
	commentstyle=\color{greencomments},
	keywordstyle=\color{bluekeywords},
	stringstyle=\color{redstrings},
	basicstyle=\ttfamily,
	tabsize=3
}
\usetheme{Execushares}
\title{A first step towards testable and maintainable code}
\subtitle{How to tame the beast that is EPiSERVER}
\author{Tommy Ivarsson}
\date{\today}

\setcounter{showSlideNumbers}{1}

\begin{document}
	\setcounter{showProgressBar}{0}
	\setcounter{showSlideNumbers}{0}

	\frame{\titlepage}

	\setcounter{framenumber}{0}
	\setcounter{showProgressBar}{1}
	\setcounter{showSlideNumbers}{1}
	\section{Testable code}
		\begin{frame}
			\frametitle{What does it mean?}
			\begin{enumerate}
				\item It is possible to write reliable and automatic tests \pause
				\item The tests are portable - They do not have external dependencies
			\end{enumerate}
		\end{frame}
		\begin{frame}
			\frametitle{Difficult to test}
			%\lstinputlisting{code-examples/file.cs}
		\end{frame}
		\begin{frame}
			\frametitle{Easy to test}
			%\lstinputlisting{code-examples/file.cs}
		\end{frame}

	\section{What are the main properties of testable code?}
		\begin{frame}
			\frametitle{Properties of a testable method} \pause
			\begin{itemize}
				\item Pure function \pause
				\begin{itemize}
					\item $f(a) = b$ \pause
					\item No side effects \pause
				\end{itemize}
				\item Independent of application state \pause
				\item Independent of environment \pause
				\item Dependencies are resolved using dependency injection
			\end{itemize}
		\end{frame}

	\section{What is dependency injection?}
		\begin{frame}
			\frametitle{Dependency injection} \pause
			\begin{exampleblock}{}
				{\large ``An injection is the passing of a dependency (a service) to a dependent object (a client).''}
				\vskip5mm
				\hspace*\fill{\small--- Wikipedia} \pause
			\end{exampleblock}
			\begin{exampleblock}{Benefits of Depedency Injection} \pause
				\begin{itemize}
					\item Code easier to test as depedencies can be mocked \pause
					\item Class dependencies listed in the constructor \pause
					\item Cleaner code which is easier to maintain
				\end{itemize}
			\end{exampleblock}
		\end{frame}
		\begin{frame}
			\frametitle{Code without dependency injection}
			\lstinputlisting{code-examples/di-no-di.cs}
		\end{frame}
		\begin{frame}
			\frametitle{Code with dependency injection}
			\lstinputlisting{code-examples/di-di.cs}
		\end{frame}

	\section{Dependency injection in EPiSERVER}
		\begin{frame}
			\frametitle{Tempting, but not good}
			\lstinputlisting{code-examples/di-servicelocator.cs} \pause
			\begin{exampleblock}{Disadvantages} \pause
				\begin{itemize}
					\item Ugly \pause
					\item Requires more code for mock setup \pause
					\item Class dependencies not listed in constructor
				\end{itemize}
			\end{exampleblock}
		\end{frame}
		\begin{frame}
			\frametitle{Correct way to get an instance}
			\lstinputlisting{code-examples/di-servicelocator-ok.cs}
		\end{frame}
		\begin{frame}
			\frametitle{Registering your own service}
			\lstinputlisting{code-examples/di-structuremap.cs}
		\end{frame}
\end{document}